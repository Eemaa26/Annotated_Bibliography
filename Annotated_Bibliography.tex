\documentclass[leavefloats]{apa6e}\usepackage[]{graphicx}\usepackage[]{color}
%% maxwidth is the original width if it is less than linewidth
%% otherwise use linewidth (to make sure the graphics do not exceed the margin)
\makeatletter
\def\maxwidth{ %
  \ifdim\Gin@nat@width>\linewidth
    \linewidth
  \else
    \Gin@nat@width
  \fi
}
\makeatother

\definecolor{fgcolor}{rgb}{0.345, 0.345, 0.345}
\newcommand{\hlnum}[1]{\textcolor[rgb]{0.686,0.059,0.569}{#1}}%
\newcommand{\hlstr}[1]{\textcolor[rgb]{0.192,0.494,0.8}{#1}}%
\newcommand{\hlcom}[1]{\textcolor[rgb]{0.678,0.584,0.686}{\textit{#1}}}%
\newcommand{\hlopt}[1]{\textcolor[rgb]{0,0,0}{#1}}%
\newcommand{\hlstd}[1]{\textcolor[rgb]{0.345,0.345,0.345}{#1}}%
\newcommand{\hlkwa}[1]{\textcolor[rgb]{0.161,0.373,0.58}{\textbf{#1}}}%
\newcommand{\hlkwb}[1]{\textcolor[rgb]{0.69,0.353,0.396}{#1}}%
\newcommand{\hlkwc}[1]{\textcolor[rgb]{0.333,0.667,0.333}{#1}}%
\newcommand{\hlkwd}[1]{\textcolor[rgb]{0.737,0.353,0.396}{\textbf{#1}}}%

\usepackage{framed}
\makeatletter
\newenvironment{kframe}{%
 \def\at@end@of@kframe{}%
 \ifinner\ifhmode%
  \def\at@end@of@kframe{\end{minipage}}%
  \begin{minipage}{\columnwidth}%
 \fi\fi%
 \def\FrameCommand##1{\hskip\@totalleftmargin \hskip-\fboxsep
 \colorbox{shadecolor}{##1}\hskip-\fboxsep
     % There is no \\@totalrightmargin, so:
     \hskip-\linewidth \hskip-\@totalleftmargin \hskip\columnwidth}%
 \MakeFramed {\advance\hsize-\width
   \@totalleftmargin\z@ \linewidth\hsize
   \@setminipage}}%
 {\par\unskip\endMakeFramed%
 \at@end@of@kframe}
\makeatother

\definecolor{shadecolor}{rgb}{.97, .97, .97}
\definecolor{messagecolor}{rgb}{0, 0, 0}
\definecolor{warningcolor}{rgb}{1, 0, 1}
\definecolor{errorcolor}{rgb}{1, 0, 0}
\newenvironment{knitrout}{}{} % an empty environment to be redefined in TeX

\usepackage{alltt}

\input{preamble.tex}

\addbibresource{refs.bib}
\IfFileExists{upquote.sty}{\usepackage{upquote}}{}
\begin{document}

\newsavebox{\mytitle}
\begin{lrbox}{\mytitle}
\begin{tabular}{c}
 % \normalfont Student Engagement with Mediational Tools in a Literacy Clinic: \\ \vspace{.5cm} \normalfont A Multimodal Examination of Clinician Designed Materials
  \normalfont Annotated Bibliography: Hypertext Markup Language as the Scholarly Writing Canvass
  %% Master of the Modes: A Multimodal Examination of Engagement with Clinician Designed Materials
\end{tabular}
\end{lrbox}

\title{\usebox\mytitle}

\shorttitle{Annotated Bibliography}
\author{Tyler W. Rinker\\
University at Buffalo/SUNY \\ \raggedright \vspace{8cm} LAI 615: Introduction to Curriculum, Instruction, and Science of Learning \\ Annotated Bibliography \\
}
\date{\today}
\maketitle




\section{\textcite{Adams2002}}


\negpar{\textcolor{darkblue}{\fullcite{Adams2002}}}



\regpar



\subsection{Summary of Research Questions \& Results/Conclusion} % A short summary of the research question and results/conclusions (75-100 words)
Advances in publishing in the twentieth century began with a move from manual processing to the use of typewriters and carbons up to the 1960s when ``illustrations were engraved on wood or copper'' (p. 29).  By the middle of the century, electronic publishing was becoming a reality.  The advent of affordable and accessible computers provided publishers with electronic files that could be readily printed with less intense manual labor.  Little improvement for the author and editor has come of these technological advances.  Publishers have assumed responsibility for much of the formerly outsourced printing.

\subsection{Evaluation of Methods \& Conclusions} % An evaluation of methods and conclusions (25-75) words.
This historical piece provides an overview of the ways in which the mechanical and social processes around publishing have changed, yet the mention of virtual content, that is content published online, is not mentioned.  This is surprising considering the 2002 publication year.

\subsection{Significance to Topic} % The contribution to your topic (why is this paper important?)
The historical overview from this chapter demonstrates how much has changed mechanically, yet how little the process has from a content perspective.  The lack of description around publishing in an online format also demonstrates the failure to keep pace with developing advances in technology.

%==========================================================================

 \section{\textcite{Anthes2012}}


\negpar{\textcolor{darkblue}{\fullcite{Anthes2012}}}



\regpar



\subsection{Summary of Research Questions \& Results/Conclusion} % A short summary of the research question and results/conclusions (75-100 words)
The latest installment of HTML (HTML5) is a simple way to mark a document in a way that a browser can interpret but provides users with interactive experiences.  This is a fundamental shift from the static document storage employed on the Internet.  HTML5 sets clear standards that makes writing in the language predictable and efficient.  Its integration with Cascading  Style Sheets (CSS) enables a flexible, powerful, and aesthetic presentation of the HTML language.  Likewise, its integration with JavaScript provides real-time computation to make user experience more interactive.  HTML5 also provides access to  text based Scalable  Vector Graphics (SVG) that yield compressible high quality representations of visuals.  HTML is a continuously evolving language and thus ``HTML is just HTML, we...dropped the number early last year. It's just continuously developed, like the browsers.'' (p. 17)

\subsection{Evaluation of Methods \& Conclusions} % An evaluation of methods and conclusions (25-75) words.
The author provides an overview of the language's current state of development, addresses areas of growth, and provides an account of its potentials.

\subsection{Significance to Topic} % The contribution to your topic (why is this paper important?)
This piece provides an understanding of how HTML is being developed and its potential uses as a publishing platform.  It also provides a general technical sense of the language that can be repurposed for a general description within our argument.

%==========================================================================

 \section{\textcite{Bazalgette2013}}


\negpar{\textcolor{darkblue}{\fullcite{Bazalgette2013}}}



\regpar



\subsection{Summary of Research Questions \& Results/Conclusion} % A short summary of the research question and results/conclusions (75-100 words)
The authors argue that the ways in which multimodal theory is taken up actually reinforces the very privilege to text it attempts to overcome.  The authors contend that the scholarly writing of many multimodal authors (predominantly Kress) gives preference to the print based modes of written language and image.  Additionally, the naming of texts as multimodal, when all text is multimodal, causes confusion in the field, with digital based modes considered to be multimodal.  This leads to a loss of attention to the ways in which various modes interact to produce meaning.

\subsection{Evaluation of Methods \& Conclusions} % An evaluation of methods and conclusions (25-75) words.
The authors provide a grounded argument for the misuses of the term multimodal.  They also provide a broad picture of the modes in context that ought to be analyzed.  The link between motion picture as a mode of special attention though was a bit weak.  The authors provide information from a specific study and age group about media preferences.  While I believe the conclusions are reasonable and grounded in more substantial work, the authors did not make this connection as strong as it could/should have been. 

\subsection{Significance to Topic} % The contribution to your topic (why is this paper important?)
This position statement provides a sense of what should be considered in a multimodal analysis.  It outlines the rich and complex nature of such studies and the insistence of high profile multimodal scholars to write about print based modes.  This may be rooted in the print based medium available to scholars in presenting research.  Print based modes are more easily discussed and demonstrated in print based medium.  They are more likely to be pursued and deemed coherent.  Conversely, HTML may afford new ways to present this rich material and encourage exploration of nonprint based modes as they would be more easily demonstrated in the interactive and flexible HTML platform.

%==========================================================================

 \section{\textcite{Bezemer2011}}


\negpar{\textcolor{darkblue}{\fullcite{Bezemer2011}}}



\regpar



\subsection{Summary of Research Questions \& Results/Conclusion} % A short summary of the research question and results/conclusions (75-100 words)
These researchers investigate the use of multimodal transcripts.  Specifically, asking how to transcribe, what to transcribe, and if researchers ought to transcribe multimodally.  This study emphasizes the re-presentation of mode in transcription, that is the act of transduction.  The authors argue that the act of transcription is never in isolation but a part of various academic processes.  These contexts frame the transcript.  The transcript is a mediator between the social situation and reader.  No mode can perfectly re-present this original event, but some are better (due to social and historical rationale) at carrying the meaning for a particular ``professional vision'' (p. 196)

\subsection{Evaluation of Methods \& Conclusions} % An evaluation of methods and conclusions (25-75) words.
The authors make a strong case for the use of, and ``correct way'' to conduct, multimodal transcription.  They note that the act of transcribing is a form of analysis.  The one criticism I have with the piece is that it purports that the transcript serves as a mediator between the event and the reader, yet the images used by the authors are grainy and difficult to read.  There is a bit of inconsistency between what the authors profess and practice, possibly as a result of their own low skill/neglect or the limitations of the medium they were publishing in.

\subsection{Significance to Topic} % The contribution to your topic (why is this paper important?)
This piece indicates the complex nature of multimodal analysis and the need to convey the event to the reader.  It also demonstrates that a poorer medium can cause reduced understanding of the meaning encoded in the modes re-presented.  This serves as an example where the crisp string vector graphics allow for better display of image and the potential for interaction.

%==========================================================================

 \section{\textcite{Dill2002}}


\negpar{\textcolor{darkblue}{\fullcite{Dill2002}}}



\regpar



\subsection{Summary of Research Questions \& Results/Conclusion} % A short summary of the research question and results/conclusions (75-100 words)
Publishers do more than simply print a text.  The majority of a publisher's work is in formatting, editing, and assuring a work meets quality, content accuracy, and legal standards.  The authors also imagine a format in which the reader will be able to interact with simulated text, specifically providing an example of a doctor practicing a heart transplant.  Publishers consider the reader's needs, the available modes, and attempt to augment the author's work in order to more efficiently and effectively deliver knowledge.  The role of publishers is needed more than ever in the new media age.

\subsection{Evaluation of Methods \& Conclusions} % An evaluation of methods and conclusions (25-75) words.
The author concludes that the role of the publisher will continue to be needed in a new media age.  They provide a sound logical argument for why the digital age will still require the expertise provided by the publishers, grounding the conclusion by presenting the overlooked, current role of publishers.

\subsection{Significance to Topic} % The contribution to your topic (why is this paper important?)
The fears of many publishers is that an electronic format could displace them.  This piece takes an approach that includes the journal as a necessary entity in the move to HTML.  This provides an argument that allows publishers to embrace a move to use HTML as the publishing platform.

%==========================================================================

 \section{\textcite{Guns2013}}


\negpar{\textcolor{darkblue}{\fullcite{Guns2013}}}



\regpar



\subsection{Summary of Research Questions \& Results/Conclusion} % A short summary of the research question and results/conclusions (75-100 words)
The authors investigate the term semantic web applied to the World Wide Web, which many argue to be a misnomer.  The authors set out to determine what is semantic about the Web.  This historical overview locates the advent of the World Wide Web in the year 1990 by CERN researchers  Tim Berners-Lee and Robert Cailliau seeking a better medium for the ``sharing of research data and results between scientists'' (p. 2174).   HTML was used to allow the computer to interface with the shared documents.  The profound implications of metadata embedding and semantic elements enable the text to be indexed and displayed in different forms with simple mark up tags.  Link typing, the ability to connect between certain HTML types, provides the strongest argument for the use of semantic to describe the Web.

\subsection{Evaluation of Methods \& Conclusions} % An evaluation of methods and conclusions (25-75) words.
The authors make convincing case for the semantic term use in semantic web.  They trace the historical roots of the Web and provide evidence as to the founders' intentions and the structure that would enable the Web to be classified as semantic.

\subsection{Significance to Topic} % The contribution to your topic (why is this paper important?)
This piece evidences that HTML was originally designed by researchers, for researchers to share data and findings.  It's not a stretch then for researchers to reclaim this birthright. 

%==========================================================================

 \section{\textcite{Knapton2013}}


\negpar{\textcolor{darkblue}{\fullcite{Knapton2013}}}



\regpar



\subsection{Summary of Research Questions \& Results/Conclusion} % A short summary of the research question and results/conclusions (75-100 words)
The author makes the argument that online sharing of audio and video data can reduce transcription demands and afford new interaction with readers.  The author argues that transcripts are not raw data but an artifact, a first step of analysis of the raw data.  The researcher faces choices about multimodal transcription including what to include, layout, and conventions.  The author points out potential loss in meaning/understanding in the process of converting from spoken to written text.  The transcription for non-verbal modes is laborious and may be unnecessary in presenting the data.  The use of URLs, and embedded video/audio would allow the researcher to demonstrate the interaction between modes more efficiently.  

\subsection{Evaluation of Methods \& Conclusions} % An evaluation of methods and conclusions (25-75) words.
The argument for online release of data is entering brave new territory.  The overview provides insightful, though unconvincing argument for releasing of audio and video data clips as part of the scholarly presentation.  The argument stems from a need to make transcription easier.  This should not be the drive of research.  The secondary argument, that the reader can gain additional insight while preserving the integrity of the data in a more understandable mode is congruent with scientific aims.  The author also makes the argument that readers will be able to make interpretations of the data directly.  While this is somewhat promising, the idea that researcher analysis may become lost serves no scientific ends.  The author makes an attempt to reconcile this potential slippery slope, but the devaluing of researcher analysis still is prevalent.

\subsection{Significance to Topic} % The contribution to your topic (why is this paper important?)
This piece begins the argument that we wish to make toward using HTML as the publishing canvass, but does not mention HTML (uses online) and by default misses the affordances of HTML beyond audio, video, and hyperlink in disseminating multimodal data.  While the main argument of these researchers starts with transcription and ends with online as a means of efficiency, we will argue that HTML better displays multimodal data and provides the reader with a more meaningful interaction with modes that better communicate the narrative of the data.

%==========================================================================

 \section{\textcite{Lie1999}}


\negpar{\textcolor{darkblue}{\fullcite{Lie1999}}}



\regpar



\subsection{Summary of Research Questions \& Results/Conclusion} % A short summary of the research question and results/conclusions (75-100 words)
HTML is a subset of Standard  Generalized  Markup  Language (SGML).  An SGML contains elements (plain text and mark up tags); within these elements is a start tag, with optional attributes, content (the text), and an ending tag.  HTML (though it strays somewhat), as an SGML, focus on describing the content (metadata) rather than declaring presentation of the content (\verb <b>  and \verb <i>  are obvious delineations from this standard).  The actually presentation formatting is done through style sheets (e.g., CSS), that instruct browsers to present the content in a manner specified, that is the HTML tags and attributes are styled based on the commands presented in the style sheet.  This system enables the content to be specified in flexible ways, for instance tags that display on a typical browser in a particular manner could be converted to audio or braille (for blind users).

\subsection{Evaluation of Methods \& Conclusions} % An evaluation of methods and conclusions (25-75) words.
The authors are clear in their discussion the ways in which HTML, CSS and XML work together to bring content to various users.  The discussion of the history, intentions, and shortfalls of HTML grant a broad conceptualization of how the language came to be and its potential for future application.

\subsection{Significance to Topic} % The contribution to your topic (why is this paper important?)
This article provides a great understanding of what HTML specifically is and is not.  It is useful information in advocating for HTML in scholarly writing because it provides a clear explanation of how HTML can serve academia and the meaning that is conveyed.

%==========================================================================

 \section{\textcite{Lynch1996}}


\negpar{\textcolor{darkblue}{\fullcite{Lynch1996}}}



\regpar



\subsection{Summary of Research Questions \& Results/Conclusion} % A short summary of the research question and results/conclusions (75-100 words)
A major issue regarding the integrity of electronic publishing is the idea that the document is not immutable.  The traditional view of print based text holds that a strength of the system is that a record is consistent across copies.  The author argues that this perception is not reality, that print based texts can be and are altered.  He argues that the fluidity of electronic text is a policy issue related to management rather than a ``weakness of technology'' (p. 135).  The authors do acknowledge that there are concerns with copying and altering that may lead to confusion and legal problems.  The doctrine of first sale ensures a library has control of a document and may make it available to whomever.  The document cannot be recalled, thus a subsequent edition is the only recourse to alter the historical record.  The author proposes the use of digital identifiers such as MD-5 checksums used with computer programs.

\subsection{Evaluation of Methods \& Conclusions} % An evaluation of methods and conclusions (25-75) words.
The author raises numerous concerns with document integrity that are built into the current print based system.  While the author does propose the use of digital identifiers, few potential solutions are generated.  Many of these questions that were not easily answered when the authored penned it are now able to be tackled.  For instance, the use of Bitcoin as a digital, unregulated, currency, provides some direction in ensuring the integrity of an electronic journal formatting system.   Electronic voting and tax filing also serve as a learning arena in which integrity is paramount.  Scholarly publishing need not answer the author's questions in isolation but draw from fields that have addressed the problem at some level.  The Digital Object Identifier and API interface discussed by \textcite{Lie1999} may also provide a solution.

\subsection{Significance to Topic} % The contribution to your topic (why is this paper important?)
This article raises a major issue of integrity that must be considered in a move to an HTML frame work.  The documents flexibility is a strength that also presents as a weakness compared to the historical integrity of a paper system.  This weakness must be addressed before HTML can be acceptable to scholars and publishers.

%==========================================================================

 \section{\textcite{Malinowski2011}}


\negpar{\textcolor{darkblue}{\fullcite{Malinowski2011}}}



\regpar



\subsection{Summary of Research Questions \& Results/Conclusion} % A short summary of the research question and results/conclusions (75-100 words)
Drawing from the works of  Saussure and Kress, this study focused on how meaning ``designed in one mode'' may affect the meaning of the constructed modes, recognizing multiplicative multimodal meaning, and the awareness of the multimodal author and its effect on generating multiplicative meaning.  The authors conclude that linguistic signs are here to stay and `` can be understood and learned and taught as a substrate of other forms of textual meaning making'' (p. 65) in much the same way that binary is the basis for all of digital technology.  \textcite{Malinowski2011} end with a recommendation that ``language learning and pedagogy must involve mastery of the fundamentals of intralinguistic systemic relations'' (p. 65) while giving value to linguistic and non-linguistic modes.


\subsection{Evaluation of Methods \& Conclusions} % An evaluation of methods and conclusions (25-75) words.
I find that the authors' substrate conclusion is weakly supported by a single visual depiction of a single participant.  Secondly, the argument uses an unfair analogy.  Binary truly is the language all of digital technology is written in.  Linguistic sign is not the underlying language of meaning.  If this were true a child, with no linguistic ability, could not learn about linguistic signs because no meaning could be established without the initial linguistic sign. The later conclusion around multiplicative sign based pedagogy is more grounded in literature and the data.

\subsection{Significance to Topic} % The contribution to your topic (why is this paper important?)
This study provides a look at the complexity in analyzing and representing a small piece of multimodally transcribed data.  This complexity can be represented linguistically, but a multiplicative use of modes provided by HTML, would enable the researcher to more partly convey intended meaning.

%==========================================================================

 \section{\textcite{Nielsen1995a}}


\negpar{\textcolor{darkblue}{\fullcite{Nielsen1995a}}}



\regpar



\subsection{Summary of Research Questions \& Results/Conclusion} % A short summary of the research question and results/conclusions (75-100 words)
Hypertext ``consists of interlinked pieces of text (or other information)'' (p. 2).  The structure of hypertext is a network of nodes (text) and edges (links); hypertext is non-sequential in contrast to traditional linear text.  The reader is given options and chooses the order and timing of information.   Frank Halasz argues that the reader only sees a particular node at any given time.  He posits that the network structure (the big picture of possible destinations) should be available to the reader at all times.  The word multimedia is often used in place of hypertext to stress the linking of various media forms, not limited to text.  

\subsection{Evaluation of Methods \& Conclusions} % An evaluation of methods and conclusions (25-75) words.
The author provides an understanding of the non-linear nature of hypertext and logically argues for the availability of the entire network to the reader.  The distinction between text and media is somewhat problematic within multimodal circles that would understand text to be more inclusive (thus no distinction between hypertext and hypermedia is necessary).  Though this argument may be due in part to the time period of the writing (1995) and the fields the author draws from.  The authors do not mention the possibility of mouse-over hypertext, though this is again due to the 1995 publication (generally, JavaScript and CSS drive a mouseover and came to popularity after the book's publication).  The author describes hypertext to be within a closed system of documents (e.g. HyperStudio) rather than a more powerful open system, linking to documents across the Internet (though this is addressed later in chapter 7).

\subsection{Significance to Topic} % The contribution to your topic (why is this paper important?)
This piece provides a definition of hypertext and provide an understanding of the nonlinearity of hypertext.  The piece states the different reader experience(s) based on the click order of hypertext.  This can be seen as a plus or a con in the argument for HTML as a publishing platform for scholarly work.

%==========================================================================

 \section{\textcite{Nielsen1995b}}


\negpar{\textcolor{darkblue}{\fullcite{Nielsen1995b}}}



\regpar



\subsection{Summary of Research Questions \& Results/Conclusion} % A short summary of the research question and results/conclusions (75-100 words)
Hypertext access in the context of the Internet over the World Wide Web (WWW or W3) refers to a networked linking of documents.  Computers are connected to servers which in turn are connected to the global network (Internet) of computers.  The standardized format for sharing text is HTML through a ``standard communication protocol called HTTP (hypertext transfer protocol).''  (p. 178)  HTML is a means of marking text ``according to its meaning'' not its ``physical display'' (p. 191).  This means browsers can interpret the tags freely, in a way that meets the specific demands of the operating system and user preferences.  The tags also have the ability to be marked with attributes that provide additional flexibility and control of the meaning of the text being marked.  A WYSIWYG (what you see is what you get) authoring tool allows the user to write a document with desired formatting without viewing the HTML tags that are inserted behind the scene.  

\subsection{Evaluation of Methods \& Conclusions} % An evaluation of methods and conclusions (25-75) words.
The author is limited in his view by the time period of the writing (before 1995), however, the basic description of the way the Internet is used to interface is still relevant.  This passage explains commonly used, though not understood, digital terms such as HTTP, HTML, WWW, WYSIWYG, browser, and Internet.

\subsection{Significance to Topic} % The contribution to your topic (why is this paper important?)
This chapter provides definitions of several key terms that must be accurately and simply described in order to make the case to use HTML, displayed on a browser, to author and disseminate scholarly work.

%==========================================================================

 \section{\textcite{Oda2002}}


\negpar{\textcolor{darkblue}{\fullcite{Oda2002}}}



\regpar



\subsection{Summary of Research Questions \& Results/Conclusion} % A short summary of the research question and results/conclusions (75-100 words)
An early eBook company, NuvoMedia, gained popularity when it was introduced the Rocket eBook in 1998.  Print publishers were not early adopters of  the eBook, however, in 2000, after Stephen King released an eBook novel, publishers quickly lined up to meet the consumer demand.  Some publishers remained reluctant to adopt electronic publications as early adoption of CD-ROM publishing in the early ages of the new media  resulted in great loss.  Publishers may also be concerned with theft of digital property or being cut out as the middle-man between author and consumer.

\subsection{Evaluation of Methods \& Conclusions} % An evaluation of methods and conclusions (25-75) words.
The author concludes that there are historical, social, and fiscal reasons a publisher may be reluctant to adopt a new media.  This is insightful and may be outside of the typical thinking a consumer may have towards adopting a new product.

\subsection{Significance to Topic} % The contribution to your topic (why is this paper important?)
While this piece pertains to book publishing, the same fiscal uncertainty is relevant to journal publishers.  This historical context provides a deeper understanding of the possible reluctance to adopt HTML as a publishing canvas.

%==========================================================================

 \section{\textcite{Peek1996a}}


\negpar{\textcolor{darkblue}{\fullcite{Peek1996a}}}



\regpar



\subsection{Summary of Research Questions \& Results/Conclusion} % A short summary of the research question and results/conclusions (75-100 words)
Journals began as a way for scholars to communicate with other scholars, that is to share works-in-progress.  One of the first uses of a journal in communication dates to the 1640s when a growing group of scholars ``who met regularly at Oxford University'' (calling themselves the Invisible College and renamed to the Royal Society) required a means to communicate beyond in person meetings.  The size of the group and limited communicational technologies led to the development of the first scholarly journal, the \emph{Philosophical Transactions of the Royal Society of London}, in 1665 (p. 5).  The original intent was to communicate and maintain an archived record.  Modern means of communication and ease of travel have altered the objectives of scholarly journals.  A major part of this change is the shift to a ``paperless environment'' (p. 6) .  

\subsection{Evaluation of Methods \& Conclusions} % An evaluation of methods and conclusions (25-75) words.
This chapter provides insights into the historical rationale for a paper system, the potentials of a paperless system, and hurdles and problems such a move would entail.  This divergent piece opens up many questions and enables the reader to gain a better perspective on the larger problem.

\subsection{Significance to Topic} % The contribution to your topic (why is this paper important?)
This chapter provides insight into the origins and original intentions of scholarly publishing.  It also will enable us to address possible concerns that using HTML (A paperless system) may generate in scholarly writing.

%==========================================================================

 \section{\textcite{Pierce2012}}


\negpar{\textcolor{darkblue}{\fullcite{Pierce2012}}}



\regpar



\subsection{Summary of Research Questions \& Results/Conclusion} % A short summary of the research question and results/conclusions (75-100 words)
This study examined the ecosystem created by and acting upon ESL participants.  The study takes a geomsemiotic stance that views the environment as a necessary part of the interaction and discourse between participants.  The author contends that power constructs between social actors may determine spatial context which in turn impacts and is impacted upon by social actors.  The study also revealed the classroom space can impact movement within the classroom.  

\subsection{Evaluation of Methods \& Conclusions} % An evaluation of methods and conclusions (25-75) words.
This trailblazing study examines classroom discourse in a complex and ecological way, considering the social actors, discourse, power, agency, and space.  It is ambitious and may set a precedent in contemplating the complexity of an educational event.  The complexity overwhelms the reader at times, keeping track of multiple participants, in multiple sites, interacting over 7 months.  Considering that the study partially examined classroom space a graphic representation of the floor plan augmented with densities or marked motion paths would have enabled the reader to better comprehend the spaces described, more so than photographic snapshots that framed the event in a narrow fashion.  Additionally, the term ``body gloss'' is used, though never explicitly defined.  The descriptions of movement, gaze, proximetrics, and other sign systems are minimally, or not grounded, within related theory \parencite[e.g.,][]{McNeil1996, McNeil2005, Norris2004, Norris2009}.

\subsection{Significance to Topic} % The contribution to your topic (why is this paper important?)
This piece demonstrates the complex questions researchers are asking and the complex situations they are attempting to model.  The piece demonstrates the constraining impact that paper oriented medium has on the ability to represent the data and convey the complex interactions to the reader in an efficient and coherent manner.

%==========================================================================

 \section{\textcite{Raman2009}}


\negpar{\textcolor{darkblue}{\fullcite{Raman2009}}}



\regpar



\subsection{Summary of Research Questions \& Results/Conclusion} % A short summary of the research question and results/conclusions (75-100 words)
This study provides a historical evolution of the World-Wide-Web, beginning as a less refined implementation of HTTP, HTML, and URLS working in conjunction.  HTML has undergone a period of market, browser driven standards to a more stable, browser independent standards.  The role of the browser is critical in providing an interface to the Web, but should not dictate how documents are constructed.  More recent implementations of HTML, CSS, JavaScript, APIs, and browsers have generated a more dependable and portable integration.

\subsection{Evaluation of Methods \& Conclusions} % An evaluation of methods and conclusions (25-75) words.
The authors provide a historical accounting of HTLM, HTTP, and URL that have struggled and evolved.  It provides an accounting of the impact of the browser glossed over, or not seen, in other historical accounts.

\subsection{Significance to Topic} % The contribution to your topic (why is this paper important?)
This piece provides an argument for the dependability of HTML as a scholarly publishing canvass.  It also highlights the role of the browser as the GUI to the content.  The piggybacking of standing structures (including the modern browser) makes development and maintaining an HTML publishing platform highly attainable.

%==========================================================================




\clearpage
\printbibliography


\end{document}
